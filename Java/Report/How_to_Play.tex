\section{How to Play}
After entering the game menu, the player can determine the number of players (1-2) and which game world to play (No.1 – No.5). Then, the player can click the start button to start the game.\\
During the game, one can see that the character(s) and the game map are in the middle of the window. On the right side of the window, there are a timer, a scoreboard, and recipes. The food orders are shown at the bottom of the window, the player can finish the order and get a score by following the recipe to cook the food and deliver it to the pickup window. If the player delivers one food that is not in the orders, he/she would get score punishment by deducting score by 10. The orders would increase during the time and the maximum number of orders is 5.\\
Both players are controlled by the keyboard.
\begin{itemize}
    \item Player1: Move \textbf{W}/\textbf{A}/\textbf{S}/\textbf{D}. Pickup food: \textbf{Q}. Place food: \textbf{E}
    \item Player2: Move \textbf{I}/\textbf{J}/\textbf{K}/\textbf{L}. Pickup food: \textbf{U}. Place food: \textbf{O}
\end{itemize}


In the game, the player(s) can get food ingredients from 7 items. There are \textbf{EggBasket}, \textbf{BreadBasket}, \textbf{CheeseBlock}, \textbf{SpinachGarden}, \textbf{PieBox}, \textbf{FruitBasket} and \textbf{TomatoBasket}.\\
Despite the fact that the \textbf{FruitBasket} can provide random fruit (apple, banana, and orange), others can only provide one kind of food which is simply shown by their names.\\
There are 4 kinds of items where food can be placed.
\begin{enumerate}
	\item \textbf{WoodPlatform} for placing at most one food.
    \item \textbf{TrashCan} for abandoning food.
    \item \textbf{ApplePieStove}, \textbf{SaladBowl}, \textbf{SandwichMaker}, and \textbf{FriedEggStove} for cooking food.
    \item \textbf{PickupWindow} for delivering food.
\end{enumerate}
When time is up, the game would enter the end page and the player could click play again to enter the menu to start a new game.\\
\quad \\
